\documentclass[a4paper,12pt]{article}

% Packages
\usepackage[margin=1in]{geometry}
\usepackage{titlesec}
\usepackage{hyperref}
\usepackage{xcolor}
\usepackage{enumitem}
\usepackage{graphicx}
\usepackage{fancyhdr}

% Header and Footer
\pagestyle{fancy}
\fancyhf{}
\fancyhead[L]{\textbf{Presentation Notes}}
\fancyhead[R]{\today}
\fancyfoot[C]{\thepage}

% Section Formatting
\titleformat{\section}{\large\bfseries\color{sectioncolor}}{\thesection.}{0.5em}{}
\titleformat{\subsection}{\normalsize\bfseries\color{subsectioncolor}}{\thesubsection.}{0.5em}{}

% Colors
\definecolor{sectioncolor}{HTML}{005f73}
\definecolor{subsectioncolor}{HTML}{0a9396}
\definecolor{notecolor}{HTML}{94d2bd}

\begin{document}

\begin{center}
    {\LARGE \textbf{Presentation Notes}} \\
    \vspace{0.2cm}
    % {\large Author: Anjali} \\
    % \vspace{0.2cm}
    {\small Date: \today}
\end{center}

General Presentation framework:
\begin{itemize}
    \item What is the topic, why does it matter for official statistics?
    \item Definitions and key methods explained simply
    \item examples
    \item contributions
    \item flowcharts, graphs, tables
    \item conclusion
\end{itemize}
\section{Section 1: Estimation of the generalization error and its uncertainty}

---

\section{Section 2:  Interpretable machine learning }
(Same format as above)
“Would you trust an ML model if you couldn’t explain its decision?”

---

\section{Section 3:  Machine learning for complex sample designs }
(Same format as above)

---

\section{Section 4: Quantitative methods for uncertainty quantification}
(Same format as above)

---

\section{Section 5 (MLops): Machine learning operations and reproducibility in official statistics}
\textbf{Key POints:}
\begin{itemize}[label=\textbullet, leftmargin=1cm]
    \item what? is mlops
    \item Why? MLOps ensures that ML systems used in official statistics are reusable, transparent and scalable
    \item also why? MLOps practices build trust in ML systems for official statistics by ensuring transparency and reproducibility
    \item concepts? version control for data and models, CI/CD pipeline
\end{itemize}

\textbf{Visual Ideas:}
\begin{itemize}[label=$\star$, leftmargin=1cm]
    \item MLOps pipeline chart
    \item explaining each step with a video or a real time example (also really depends on how familiar they are with this)
    \item example of dealing with bias (the formula)
    \item visually showing the way ahead? 
    \item the connection between Interpretability and fairness of ML
    \item A messy vs a clean process (good way to understand the importance better)
    \item also like what you have now + the additions we are trying to get
    \item results first and then the pipeline description
    \item the git repository
    \item talking about mlops best practices to be sure of the internal working standards and handling possible future problems (maybe a document for this would be enough though)
    \item extensions, couldera, data leaks (not sure how much depth to go into these)
\end{itemize}


---

\section{Section 6: Fairness and Bias Auditing}
\textbf{Key Points:}
\begin{itemize}[label=\textbullet, leftmargin=1cm]
    \item techinical + social and societal challenges of ML in public sector
    \item what are the problems? ans: shifts in decision making responsibility and immense technical stability
    \item protected attributes
    \item measurement errors can also affect model training
    \item group, subgroup and individual fairness (concepts, and challenge)
    \item prediction and decision step (talking about how crucial the prediction step in official statistics is)
    \item The fairness of ADM systems starts way before decisions are made. Every step in the data pipeline - from designing surveys to cleaning and processing data - contributes to the fairness (or unfairness) of the final sustem
    \item Some effects may have bigger impact than others
    \item Two steps of ADM where prediction step directly affects the decision step
    \item integrating fairness aspects into existing quality criterion (contributions)
    \item the way ahead with the possible new opportunities (contribution)
\end{itemize}

\textbf{Visual Ideas:}
\begin{itemize}[label=$\star$, leftmargin=1cm]
    \item Example: Bias in a loan approval system 
    \item Data collection -> Cleaning -> Training -> Prediction -> Decision, with bias sources highlighted at each stage.
    \item example of dealing with bias (the formula)
    \item visually showing the way ahead? 
    \item the connection between Interpretability and fairness of ML
\end{itemize}


\textbf{Contributions}
\begin{enumerate}[label=C\arabic*:, leftmargin=1cm]
    \item we propose an extension of the “Quality Framework for Statistical Algorithms” (QF4SA; Yung et al. 2022), with which fairness considerations can be embedded in existing quality guidelines
    \item suggesting, Data from official statistics can be used as benchmark data for fairness evaluations, both in comparison to other data and for ML models themselves.
    
\end{enumerate}


\begin{itemize}
    \item are we talking to different sets of people, or is it one audience
    \item what is their familirarity wiht the existing mehtods 
    \item how much stat/math/ml/code/python/r/mlops do they know? even like github?
    \item atleast for the coded parts, maybe a document with the tutorial would be very benefical. or a recorded one because this might possibly be a big struggle
    \item questions: what? why? what has been working so far (objective 1) ? how are we helping this to make it better (objective 2)? how to integrate this all and go ahead? 
    \item visual diagrams of the pipeline or the problem + real world synthetic official statistics examples
    \item not go too theoretical, but explain it more visually 
    \item 
\end{itemize}
\end{document}